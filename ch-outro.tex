\documentclass[main.tex]{subfiles}
\begin{document}
\chapter{Conclusion}

\epigraph{So, that's it?}{Justin Goodman}

Yes, that is it. That is Discrete Mathematics. You did it, congratulations!

\section{Closing Remarks}

The fun does not stop here -- there are many more topics in discrete mathematics! This book is geared towards students in an introductory proof course. One can also go arbitrarily deep in any topic listed in this book. Mathematics is beautiful in this sense -- there is no endpoint. There will always be a bigger number. There will always be a more difficult unsolved problem in mathematics. There is no ending in sight, but this infinite journey is worth more than the finale.

\section{Tips}

Here is a compiled list of tips. There are tips for each topic related to taking a course corresponding to this book. Each list is in no particular order.

\subsection{Studying}

\begin{itemize} %[wide, labelwidth=!, labelindent=0cm]
	\item Take advantage of the Spacing Effect -- spread your studying out over time.
	\item Utilize the Pomodoro Technique. Repeat the following process: study for 25 minutes, take a 5 minute break, repeat 4 times, take a longer 15-30 minute break.
	\item Teach your peers about what you are studying -- if you can teach someone a topic, then you understand that topic.
\end{itemize}

\subsection{Assignments}

\begin{itemize} %[wide, labelwidth=!, labelindent=0cm]
	\item Start early.
	\item Work with others.
	\item Struggle until you understand.
	\item Seek help, not solutions.
\end{itemize}

\subsection{Exams}

\begin{itemize} %[wide, labelwidth=!, labelindent=0cm]
	\item Get plenty of sleep before the exam. 8 hours of sleep with 2 hours of studying is significantly better than 2 hours of sleep with 8 hours of studying.
	\item Eat food, drink water, exercise, and use the bathroom before the exam. Avoid drugs.
	\item Be aware of each concept, even if you forgot to study it. If on the exam you come across something you are rusty with, then write down as much as you know about the problem.
	\item Write neatly.
	\item Often, your first instinct answer is the correct answer. Stay away from over-analyzing a question.
	\item During the exam when you have time, re-read each question and make sure you are actually answering the question.
	\item If you read a problem and the answer does not immediately jump out at you, or if you get stuck on a problem, mark it and move on. Sometimes other exam questions will give you the insight you need to go back and answer the marked problem.
	\item Relax.
	\item Plan out how you want to attack this exam. % If, for example, the exam is 2 hours and has 6 questions, then you have 20 minutes per question. Maybe your plan is to spend 2 minutes reading a question, 10 minutes answering a question, and 8 minutes checking your work (repeat 6 times). Maybe you are more comfortable on topic A than topic B, so you want to allot more time to topic B but save it for last. You have to find a system that works for you, but come to the exam knowing your plan.
	\item \textbf{Remember the big picture of life.}
\end{itemize}

\section{Summary}

\begin{itemize}
	\item This is a long and difficult course
	\item Be proud of yourself
	\item You got this
\end{itemize}

\end{document}
