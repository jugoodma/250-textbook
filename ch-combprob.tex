\documentclass[main.tex]{subfiles}
\begin{document}
\chapter{Combinatorics and Probability}

\epigraph{What?!}{Lil Jon}

\minitoc

\section{Introduction}

This is \textit{Discrete} Mathematics, so we will be focusing on \textit{discrete} probability.

Let's put some more text here for now

\section{Combinatorics}

Imagine you are a software developer for a large company, \textit{Macrosoft}. Your manager assigned you to a project that needs test cases written. Your test cases must cover all execution paths. At first this seems like a daunting task, however you took discrete mathematics and bleh bleh 

(just say a story that shows that counting techniques are useful for creating test cases)

\begin{defn}[Combinatorics\index{Combinatorics}]
	
\end{defn}

\begin{example}
	A problem here
\end{example}

\begin{defn}[Permutations\index{Combinatorics!Permutations}]
	
\end{defn}

\begin{example}
	A problem here
\end{example}

\begin{defn}[Combinations\index{Combinatorics!Combinations}]
	
\end{defn}

\begin{example}
	A problem here
\end{example}

A mini-discussion on the differences between ordering and replacement.

Insert that one table here.

\section{Pigeonhole Principle}

\begin{thm}[The Pigeonhole Principle]
	
\end{thm}

\begin{thm}[Generalized Pigeonhole Principle]
	
\end{thm}

\section{Discrete Probability}

Intro text here

\begin{defn}[Probability\index{Probability}]
	The likelihood (percent chance) that an event occurs. Probabilities of every possible event should add to 1
\end{defn}

\begin{example}
	
\end{example}

Something else here

\begin{example}
	Harder example
\end{example}

Joint/Disjoint

\begin{defn}[Joint Probability\index{Probability!Joint}]
	
\end{defn}

\begin{defn}[Independence\index{Probability!Independence}]
	
\end{defn}

Some discussion on conditional probability

\begin{defn}[Conditional Probability\index{Probability!Conditional}]
	
\end{defn}

\begin{defn}[Conditional Probability\index{Probability!Conditional}]
	
\end{defn}

Probably talk about 

\section{Basic Statistics\index{Statistics}}

Statistics are an important application of probability. Statistics give us a way to mathematically model probabilities of big events/experiments. We cover three fundamental statistics principles, along with a few related topics.

\begin{defn}[Expected Value\index{Statistics!Expected Value}]
	\(\mathbb{E}[X] = \mu_X = \sum_{x}^{} x \cdot \mathrm{Pr}(X = x)\). This is a generalization of the \textit{arithmetic mean}, which is defined in scenarios where outcomes are equally likely. If we denote \(x_1,\ x_2,\ x_3,\ \cdots,\ x_n\) to be the values outputted by \(X\), and \(p_1,\ p_2,\ p_3,\ \cdots,\ p_n\) to be their respective probabilities, then \(\mathbb{E}[X] = \sum_{i=1}^{n} x_i p_i\)
\end{defn}

Sometimes students confuse \textit{expected value} with \textit{average}. First, the ``average'' is not well-defined -- it could refer to the mean, median, or mode (typically it refers to the mean). Second, as stated in the definition, the expected value is a generalized \textit{mean}.

\begin{defn}[Variance\index{Statistics!Variance}]
	
\end{defn}

\begin{defn}[Standard Deviation\index{Statistics!Standard Deviation}]
	
\end{defn}

\section{Summary}

\begin{itemize}
	\item 1
	\item 2
	\item 3
\end{itemize}

\section{Practice}

\begin{enumerate}
	\item q
	\item 2 of 20 light-bulbs are defective. You select 2 light-bulbs at random. What is the probability that neither bulb selected is defective?
	% (18/20) * (17/19)
	\item q
\end{enumerate}
\end{document}